\documentclass[11pt,a4paper]{article}
\usepackage[utf8]{inputenc}
\usepackage[T1]{fontenc}
\usepackage[french]{babel}
\usepackage{geometry}
\usepackage{graphicx}
\usepackage{enumitem}
\usepackage{float}
\usepackage{tikz}
\usetikzlibrary{shapes,arrows,positioning}

\geometry{hmargin=2.5cm,vmargin=2.5cm}

\title{Résolution des Exercices 1, 2 et 3}
\author{}
\date{}

\begin{document}

\section*{Exercice 1 : Composants d'une tour d'ordinateur de bureau}

Les principaux composants que l'on trouve dans une tour d'ordinateur de bureau sont :

\begin{itemize}
    \item \textbf{La carte mère (Motherboard)} : Le circuit imprimé principal qui connecte tous les composants.
    \item \textbf{Le processeur (CPU)} : Le cerveau de l'ordinateur qui exécute les instructions.
    \item \textbf{Le ventirad (Système de refroidissement)} : Dissipateur thermique et ventilateur pour le processeur.
    \item \textbf{La mémoire vive (RAM)} : Stocke temporairement les données en cours d'utilisation.
    \item \textbf{Le disque dur (HDD) ou disque SSD} : Pour le stockage permanent des données et du système d'exploitation.
    \item \textbf{Le bloc d'alimentation (PSU)} : Fournit l'électricité aux composants.
    \item \textbf{La carte graphique (GPU)} : Gère l'affichage à l'écran (peut être intégrée au CPU ou dédiée).
    \item \textbf{Le lecteur optique} (Optionnel, de plus en plus rare).
    \item \textbf{La carte réseau / Carte Wi-Fi} (Souvent intégrée à la carte mère).
    \item \textbf{La carte son} (Souvent intégrée à la carte mère).
\end{itemize}

\begin{figure}[H]
\centering
\includegraphics[width=0.8\textwidth]{image_ex1_pc_components.jpg}
\caption{Vue interne des composants d'un ordinateur}
\textit{Source : Image générée par IA (Amazon Titan Image Generator v2)}
\end{figure}

% Schéma simpliste d'une tour (Optionnel, mais visuellement aidant)
\begin{figure}[H]
\centering
\begin{tikzpicture}[node distance=1cm, auto, scale=0.8, every node/.style={transform shape}]
    % Boitier
    \draw[thick] (0,0) rectangle (10,12);
    \node[above] at (5,12) {Tour d'ordinateur};
    
    % Alimentation
    \node[draw, fill=gray!20, minimum width=3cm, minimum height=2cm] (psu) at (2,10.5) {Alimentation};
    
    % Carte Mère
    \node[draw, fill=green!10, minimum width=6cm, minimum height=8cm] (cm) at (5,5) {};
    \node[anchor=north] at (5,9) {Carte Mère};
    
    % CPU
    \node[draw, fill=blue!20, minimum size=1.5cm] (cpu) at (5,7) {CPU + Ventirad};
    
    % RAM
    \node[draw, fill=yellow!20, rotate=90, minimum width=2cm, minimum height=0.5cm] (ram) at (7,7) {RAM};
    \node[draw, fill=yellow!20, rotate=90, minimum width=2cm, minimum height=0.5cm] at (7.6,7) {RAM};
    
    % GPU
    \node[draw, fill=red!20, minimum width=4cm, minimum height=1cm] (gpu) at (5,3) {Carte Graphique (PCIe)};
    
    % Stockage
    \node[draw, fill=orange!20, minimum width=2.5cm, minimum height=1.5cm] (hdd) at (8.5,2) {HDD/SSD};
    
    % Lecteur Optique
    \node[draw, fill=cyan!10, minimum width=2.5cm, minimum height=1cm] (odd) at (8.5,10.5) {Lecteur Optique};

\end{tikzpicture}
\caption{Schéma simplifié de l'agencement interne d'une tour PC}
\end{figure}

\section*{Exercice 2 : Analyse du scénario et périphériques}

Basé sur le scénario de l'étudiant de l'ISTEAH, voici les périphériques identifiés et leurs types :

\begin{figure}[H]
\centering
\includegraphics[width=0.8\textwidth]{image_ex2_peripherals.jpg}
\caption{Illustration du poste de travail et des périphériques}
\textit{Source : Image générée par IA (Amazon Titan Image Generator v2)}
\end{figure}

\begin{center}
\renewcommand{\arraystretch}{1.5}
\begin{tabular}{|l|c|p{8cm}|}
\hline
\textbf{Périphérique} & \textbf{Type} & \textbf{Justification dans le texte} \\
\hline
\textbf{Clavier} & Entrée & "Il saisit son texte à l’aide du clavier" \\
\hline
\textbf{Souris} & Entrée & "utilise la souris pour corriger certaines parties" \\
\hline
\textbf{Écran} & Sortie & "Il regarde alors une capsule vidéo" (Impliqué par le visionnage) \\
\hline
\textbf{Écouteurs} & Sortie & "il met ses écouteurs" \\
\hline
\textbf{Clé USB} & Entrée / Sortie & "copie la partie déjà réalisée du devoir sur une clé USB" \\
\hline
\textbf{Imprimante} & Sortie & "il imprime la partie déjà faite de son devoir" \\
\hline
\end{tabular}
\end{center}

\newpage

\section*{Exercice 3 : Rôles des composants numériques (Architecture de Von Neumann)}

\begin{figure}[H]
\centering
\begin{tikzpicture}[
    block/.style={rectangle, draw, fill=white, text width=2.5cm, text centered, rounded corners, minimum height=1.2cm},
    line/.style={draw, -latex'},
    cloud/.style={draw, ellipse, fill=red!20, node distance=3cm, minimum height=2em}
]

% CPU Components Layout
\node[block, fill=blue!10] (pc) {Compteur de Programme (PC)};
\node[block, fill=blue!10, right=1cm of pc] (mar) {Registre d'Adresse (MAR)};
\node[block, fill=green!10, right=1cm of mar] (mem) {Mémoire Principale (RAM)};

\node[block, fill=yellow!10, below=1cm of pc] (ir) {Registre d'Instructions (IR)};
\node[block, fill=orange!10, below=1cm of mar] (mdr) {Registre de Données (MDR)};

\node[block, fill=purple!10, below=1cm of ir] (cu) {Unité de Contrôle};
\node[block, fill=red!10, below=1cm of mdr] (acc) {Accumulateur (AC)};

\node[block, fill=lightgray, below=1cm of acc] (alu) {Unité Arithmétique et Logique (UAL)};

% Buses (Generic)
\draw[line, thick] (pc) -- (mar);
\draw[line, thick] (mar) -- (mem);
\draw[line, thick] (mem) -- (mdr);
\draw[line, thick] (mdr) -- (ir);
\draw[line, thick] (mdr) -- (acc);
\draw[line, thick] (acc) -- (alu);
\draw[line, thick] (cu) -| (ir);

% Input/Output placeholders
\node[block, fill=gray!10, left=1cm of pc] (inpr) {Registre d'Entrée (INPR)};
\node[block, fill=gray!10, left=1cm of alu] (outr) {Registre de Sortie (OUTR)};

% Connections
\draw[line, dashed] (inpr) -- (mdr); % Simplified flow
\draw[line, dashed] (alu) -- (outr); % Simplified flow

\end{tikzpicture}
\caption{Schéma conceptuel de l'Architecture de Von Neumann}
\end{figure}

Voici les définitions et rôles des composants numériques dans l'architecture de base d'un ordinateur :

\begin{description}
    \item[Compteur de programme (PC)] \hfill \\
    C'est un registre qui contient l'adresse mémoire de la \textit{prochaine} instruction à exécuter par le processeur. Il est incrémenté automatiquement après chaque lecture d'instruction.

    \item[Registre de données (MDR)] \hfill \\
    Ce registre stocke temporairement les données qui viennent d'être lues de la mémoire ou celles qui sont prêtes à être écrites dans la mémoire. Il agit comme un tampon entre le processeur et la mémoire centrale.

    \item[Accumulateur (AC)] \hfill \\
    C'est un registre spécial de l'unité arithmétique et logique ((UAL) utilisé pour stocker les résultats intermédiaires des opérations arithmétiques et logiques.

    \item[Registre d'instructions (IR)] \hfill \\
    Ce registre contient l'instruction qui est \textit{actuellement} en cours d'exécution. Le processeur décode le contenu de ce registre pour savoir quelle opération effectuer.

    \item[Registre tampon] \hfill \\
    Un registre utilisé pour stocker temporairement des données lors de leur transfert entre deux unités fonctionnelles fonctionnant à des vitesses différentes (par exemple, entre le processeur et un périphérique).

    \item[Registre de sortie (OUTR)] \hfill \\
    Ce registre conserve les données traitées par le processeur avant qu'elles ne soient envoyées vers un périphérique de sortie (comme un écran ou une imprimante).

    \item[Registre d'entrée (INPR)] \hfill \\
    Ce registre reçoit et stocke temporairement les données provenant d'un périphérique d'entrée (comme un clavier) avant qu'elles ne soient traitées par le processeur.

    \item[Registre d'adresse (MAR)] \hfill \\
    Ce registre contient l'adresse de l'emplacement mémoire auquel le processeur veut accéder, que ce soit pour lire une instruction/donnée ou pour écrire une donnée.
\end{description}

\end{document}
