\documentclass[11pt,a4paper]{article}
\usepackage[utf8]{inputenc}
\usepackage[T1]{fontenc}
\usepackage[french]{babel}
\usepackage{geometry}
\usepackage{graphicx}
\usepackage{enumitem}
\usepackage{float}
\usepackage{tikz}
\usepackage{svg}
\usepackage{amsmath}
\usepackage{amssymb}
\usepackage{fancyhdr}
\usepackage{caption}

\usetikzlibrary{shapes,arrows,positioning,decorations.pathreplacing,calc}
\usepackage{multirow}

\newcommand*\circled[1]{\tikz[baseline=(char.base)]{%
  \node[shape=circle, draw, inner sep=1pt, font=\scriptsize\bfseries] (char) {#1};}}

\geometry{hmargin=2.5cm,vmargin=2.5cm}

% Configuration de la page
\geometry{margin=2.5cm}
\pagestyle{fancy}
\fancyhf{}
\fancyhead[L]{INF101 - Devoir 1}
\fancyhead[R]{Hiver 2026}
\fancyfoot[C]{\thepage}


\begin{document}
   \begin{titlepage}
    \centering
    
    % Logo
    \includegraphics[width=0.3\textwidth]{isteah.jpg}
    
    \vspace{2cm}
    
    {\LARGE\bfseries INF101\\Introduction aux outils informatiques\par}
    
    \vspace{1.5cm}
    
    {\Huge\bfseries Devoir 1\par}
    
    \vspace{2cm}
    
    {\Large Présenté par:\par}
    
    \vspace{0.5cm}
    
    {\large
    JOSEPH Samuel Jonathan\\
    DELIA Yves Anzt-ky\par}
    
    \vfill
    
    {\large Hiver 2026\par}
    
\end{titlepage}

\newpage

%exercice fait par S
\section*{Exercice 1 : Composants d'une tour d'ordinateur de bureau}

Les principaux composants que l'on trouve dans une tour d'ordinateur de bureau sont :

\begin{itemize}
    \item \textbf{La carte mère (Motherboard)} : Le circuit imprimé principal qui connecte tous les composants.
    \item \textbf{Le processeur (CPU)} : Le cerveau de l'ordinateur qui exécute les instructions.
    \item \textbf{Le ventirad (Système de refroidissement)} : Dissipateur thermique et ventilateur pour le processeur.
    \item \textbf{La mémoire vive (RAM)} : Stocke temporairement les données en cours d'utilisation.
    \item \textbf{Le disque dur (HDD) ou disque SSD} : Pour le stockage permanent des données et du système d'exploitation.
    \item \textbf{Le bloc d'alimentation (PSU)} : Fournit l'électricité aux composants.
    \item \textbf{La carte graphique (GPU)} : Gère l'affichage à l'écran (peut être intégrée au CPU ou dédiée).
    \item \textbf{Le lecteur optique} (Optionnel, de plus en plus rare).
    \item \textbf{La carte réseau / Carte Wi-Fi} (Souvent intégrée à la carte mère).
    \item \textbf{La carte son} (Souvent intégrée à la carte mère).
\end{itemize}

\begin{figure}[H]
\centering
\includegraphics[width=0.8\textwidth]{image_ex1_pc_components.jpg}
\caption{Vue interne des composants d'un ordinateur}
\textit{Source : Image générée par IA (Amazon Titan Image Generator v2)}
\end{figure}

% Schéma simpliste d'une tour (Optionnel, mais visuellement aidant)
\begin{figure}[H]
\centering
\begin{tikzpicture}[node distance=1cm, auto, scale=0.8, every node/.style={transform shape}]
    % Boitier
    \draw[thick] (0,0) rectangle (10,12);
    \node[above] at (5,12) {Tour d'ordinateur};
    
    % Alimentation
    \node[draw, fill=gray!20, minimum width=3cm, minimum height=2cm] (psu) at (2,10.5) {Alimentation};
    
    % Carte Mère
    \node[draw, fill=green!10, minimum width=6cm, minimum height=8cm] (cm) at (5,5) {};
    \node[anchor=north] at (5,9) {Carte Mère};
    
    % CPU
    \node[draw, fill=blue!20, minimum size=1.5cm] (cpu) at (5,7) {CPU + Ventirad};
    
    % RAM
    \node[draw, fill=yellow!20, rotate=90, minimum width=2cm, minimum height=0.5cm] (ram) at (7,7) {RAM};
    \node[draw, fill=yellow!20, rotate=90, minimum width=2cm, minimum height=0.5cm] at (7.6,7) {RAM};
    
    % GPU
    \node[draw, fill=red!20, minimum width=4cm, minimum height=1cm] (gpu) at (5,3) {Carte Graphique (PCIe)};
    
    % Stockage
    \node[draw, fill=orange!20, minimum width=2.5cm, minimum height=1.5cm] (hdd) at (8.5,2) {HDD/SSD};
    
    % Lecteur Optique
    \node[draw, fill=cyan!10, minimum width=2.5cm, minimum height=1cm] (odd) at (8.5,10.5) {Lecteur Optique};

\end{tikzpicture}
\caption{Schéma simplifié de l'agencement interne d'une tour PC}
\end{figure}

\section*{Exercice 2 : Analyse du scénario et périphériques}

Basé sur le scénario de l'étudiant de l'ISTEAH, voici les périphériques identifiés et leurs types :

\begin{figure}[H]
\centering
\includegraphics[width=0.8\textwidth]{image_ex2_peripherals.jpg}
\caption{Illustration du poste de travail et des périphériques}
\textit{Source : Image générée par IA (Amazon Titan Image Generator v2)}
\end{figure}

\begin{center}
\renewcommand{\arraystretch}{1.5}
\begin{tabular}{|l|c|p{8cm}|}
\hline
\textbf{Périphérique} & \textbf{Type} & \textbf{Justification dans le texte} \\
\hline
\textbf{Clavier} & Entrée & "Il saisit son texte à l’aide du clavier" \\
\hline
\textbf{Souris} & Entrée & "utilise la souris pour corriger certaines parties" \\
\hline
\textbf{Écran} & Sortie & "Il regarde alors une capsule vidéo" (Impliqué par le visionnage) \\
\hline
\textbf{Écouteurs} & Sortie & "il met ses écouteurs" \\
\hline
\textbf{Clé USB} & Entrée / Sortie & "copie la partie déjà réalisée du devoir sur une clé USB" \\
\hline
\textbf{Imprimante} & Sortie & "il imprime la partie déjà faite de son devoir" \\
\hline
\end{tabular}
\end{center}

\newpage

\section*{Exercice 3 : Rôles des composants numériques (Architecture de Von Neumann)}

\begin{figure}[H]
\centering
\begin{tikzpicture}[
    block/.style={rectangle, draw, fill=white, text width=2.5cm, text centered, rounded corners, minimum height=1.2cm},
    line/.style={draw, -latex'},
    cloud/.style={draw, ellipse, fill=red!20, node distance=3cm, minimum height=2em}
]

% CPU Components Layout
\node[block, fill=blue!10] (pc) {Compteur de Programme (PC)};
\node[block, fill=blue!10, right=1cm of pc] (mar) {Registre d'Adresse (MAR)};
\node[block, fill=green!10, right=1cm of mar] (mem) {Mémoire Principale (RAM)};

\node[block, fill=yellow!10, below=1cm of pc] (ir) {Registre d'Instructions (IR)};
\node[block, fill=orange!10, below=1cm of mar] (mdr) {Registre de Données (MDR)};

\node[block, fill=purple!10, below=1cm of ir] (cu) {Unité de Contrôle};
\node[block, fill=red!10, below=1cm of mdr] (acc) {Accumulateur (AC)};

\node[block, fill=lightgray, below=1cm of acc] (alu) {Unité Arithmétique et Logique (UAL)};

% Buses (Generic)
\draw[line, thick] (pc) -- (mar);
\draw[line, thick] (mar) -- (mem);
\draw[line, thick] (mem) -- (mdr);
\draw[line, thick] (mdr) -- (ir);
\draw[line, thick] (mdr) -- (acc);
\draw[line, thick] (acc) -- (alu);
\draw[line, thick] (cu) -| (ir);

% Input/Output placeholders
\node[block, fill=gray!10, left=1cm of pc] (inpr) {Registre d'Entrée (INPR)};
\node[block, fill=gray!10, left=1cm of alu] (outr) {Registre de Sortie (OUTR)};

% Connections
\draw[line, dashed] (inpr) -- (mdr); % Simplified flow
\draw[line, dashed] (alu) -- (outr); % Simplified flow

\end{tikzpicture}
\caption{Schéma conceptuel de l'Architecture de Von Neumann}
\end{figure}

Voici les définitions et rôles des composants numériques dans l'architecture de base d'un ordinateur :

\begin{description}
    \item[Compteur de programme (PC)] \hfill \\
    C'est un registre qui contient l'adresse mémoire de la \textit{prochaine} instruction à exécuter par le processeur. Il est incrémenté automatiquement après chaque lecture d'instruction.

    \item[Registre de données (MDR)] \hfill \\
    Ce registre stocke temporairement les données qui viennent d'être lues de la mémoire ou celles qui sont prêtes à être écrites dans la mémoire. Il agit comme un tampon entre le processeur et la mémoire centrale.

    \item[Accumulateur (AC)] \hfill \\
    C'est un registre spécial de l'unité arithmétique et logique ((UAL) utilisé pour stocker les résultats intermédiaires des opérations arithmétiques et logiques.

    \item[Registre d'instructions (IR)] \hfill \\
    Ce registre contient l'instruction qui est \textit{actuellement} en cours d'exécution. Le processeur décode le contenu de ce registre pour savoir quelle opération effectuer.

    \item[Registre tampon] \hfill \\
    Un registre utilisé pour stocker temporairement des données lors de leur transfert entre deux unités fonctionnelles fonctionnant à des vitesses différentes (par exemple, entre le processeur et un périphérique).

    \item[Registre de sortie (OUTR)] \hfill \\
    Ce registre conserve les données traitées par le processeur avant qu'elles ne soient envoyées vers un périphérique de sortie (comme un écran ou une imprimante).

    \item[Registre d'entrée (INPR)] \hfill \\
    Ce registre reçoit et stocke temporairement les données provenant d'un périphérique d'entrée (comme un clavier) avant qu'elles ne soient traitées par le processeur.

    \item[Registre d'adresse (MAR)] \hfill \\
    Ce registre contient l'adresse de l'emplacement mémoire auquel le processeur veut accéder, que ce soit pour lire une instruction/donnée ou pour écrire une donnée.
\end{description}

\section*{Exercice 4 : Convertissez les nombres décimaux suivants en bases indiquées}

\subsubsection*{a. 265 en Binaire}

On divise par 2 successivement :

\begin{itemize}
   \item \textbf{$265 \div 2 = 132$} , reste 1
   \item \textbf{$132 \div 2 = 66$} , reste 0
   \item \textbf{$66 \div 2 = 33$} , reste 0
   \item \textbf{$33 \div 2 = 16$} , reste 1
     \item \textbf{$16 \div 2 = 8$} , reste 0
   \item \textbf{$8 \div 2 = 4$} , reste 0
   \item \textbf{$4 \div 2 = 2$} , reste 0
     \item \textbf{$2 \div 2 = 1$} , reste 0
   \item \textbf{$1 \div 2 = 0$} , reste 1
\end{itemize}

\subsubsection*{Le resultat est donc $(265)_{10} = \mathbf{100001001_2}$ , en faisant la notation du bas vers le haut} 

\subsubsection*{ b. 839 à octal}

On divise par 8 successivement :

\begin{itemize}
   \item \textbf{$839 \div 8 = 104$} , reste 7
   \item \textbf{$104 \div 8 = 13$} , reste 0
   \item \textbf{$13 \div 8 = 1$} , reste 5
   \item \textbf{$1 \div 8 = 0$} , reste 1
\end{itemize}

\subsubsection*{Le resultat est donc $(839)_{10} = \mathbf{1507_8}$ en faisant  la notation du bas vers le haut}

\subsubsection*{c. 572 en hexadécimal}


\begin{itemize}
   \item \textbf{$572 \div 16 = 35$}, reste 12 (ce qui correspond à C)
   \item \textbf{$35 \div 16 = 2$} , reste 3
   \item \textbf{$2 \div 16 = 0$} , reste 2
\end{itemize}

\subsubsection*{$(572)_{10} = \mathbf{23C_{16}}$ en faisant  la notation du bas vers le haut}

\newpage

\section*{Exercice 5 : Cycle d'instruction}

Un cycle d'instruction est la séquence d'opérations effectuées par le processeur pour traiter une seule instruction. Il se décompose en trois phases : \textbf{Recherche} (Fetch), \textbf{Décodage} (Decode) et \textbf{Exécution} (Execute).

\medskip
Prenons l'exemple suivant : l'instruction \texttt{ADD 200} est stockée à l'adresse mémoire \texttt{100}, et la case mémoire \texttt{200} contient la valeur \texttt{45}.

\begin{figure}[H]
\centering
\begin{tikzpicture}[
    phase/.style={rectangle, draw, fill=blue!15, text width=3.5cm, text centered, rounded corners, minimum height=1.2cm, font=\large\bfseries},
    arrow/.style={-latex', very thick, blue!60}
]
\node[phase] (fetch) at (0,0) {Recherche\\(Fetch)};
\node[phase] (decode) at (5.5,0) {Décodage\\(Decode)};
\node[phase] (execute) at (11,0) {Exécution\\(Execute)};

\draw[arrow] (fetch) -- (decode);
\draw[arrow] (decode) -- (execute);
\draw[arrow, dashed] (execute.south) -- ++(0,-0.8) -| (fetch.south) node[pos=0.25, below, font=\small\itshape] {Cycle suivant};
\end{tikzpicture}
\caption{Les trois phases du cycle d'instruction}
\end{figure}

\subsection*{Phase 1 -- Recherche (Fetch)}
Le processeur va chercher l'instruction en mémoire :
\begin{enumerate}
    \item Le contenu du compteur de programme (PC) est copié dans le \textbf{registre d'adresse} (MAR) : MAR $\leftarrow$ \texttt{100}
    \item La mémoire renvoie le contenu de cette adresse dans le \textbf{registre de données} (MDR) : MDR $\leftarrow$ \texttt{ADD 200}
    \item Le contenu du MDR est transféré dans le \textbf{registre d'instruction} (IR) : IR $\leftarrow$ \texttt{ADD 200}
    \item Le compteur de programme est incrémenté : PC $\leftarrow$ \texttt{101}
\end{enumerate}

\subsection*{Phase 2 -- Décodage (Decode)}
\begin{enumerate}
    \item L'unité de contrôle analyse le contenu de l'IR et identifie l'opération (\texttt{ADD}) et l'adresse de l'opérande (\texttt{200}).
\end{enumerate}

\subsection*{Phase 3 -- Exécution (Execute)}
\begin{enumerate}
    \item L'adresse de l'opérande est placée dans le \textbf{registre d'adresse} : MAR $\leftarrow$ \texttt{200}
    \item La valeur à cette adresse est lue dans le \textbf{registre de données} : MDR $\leftarrow$ \texttt{45}
    \item L'opération est effectuée dans le \textbf{registre du processeur} (accumulateur) : AC $\leftarrow$ AC + 45
\end{enumerate}

\subsection*{État des registres à chaque phase}

\begin{center}
\small
\renewcommand{\arraystretch}{1.4}
\begin{tabular}{|l|c|c|c|c|c|}
\hline
\textbf{Phase} & \textbf{Reg. adr.} & \textbf{Reg. mém.} & \textbf{Reg. instr.} & \textbf{Reg. donn.} & \textbf{Reg. proc.} \\
 & (MAR) & (Mémoire) & (IR) & (MDR) & (AC) \\
\hline
Recherche & 100 & Mém[100] lu & ADD 200 & ADD 200 & -- \\
\hline
Décodage & -- & -- & ADD 200 & -- & -- \\
\hline
Exécution & 200 & Mém[200] lu & ADD 200 & 45 & AC + 45 \\
\hline
\end{tabular}
\end{center}

\begin{figure}[H]
\centering
\begin{tikzpicture}[
    reg/.style={rectangle, draw, fill=yellow!10, minimum width=2.8cm, minimum height=1cm, align=center, font=\footnotesize},
    mem/.style={rectangle, draw, fill=green!10, minimum width=2.8cm, minimum height=1cm, align=center, font=\footnotesize},
    label/.style={font=\footnotesize\bfseries, above},
    arrow/.style={-latex', thick},
    data/.style={font=\footnotesize\ttfamily, red!70!black}
]

% Registers
\node[reg] (pc) at (0,0) {PC\\= \texttt{100}};
\node[label] at (pc.north) {Compteur Prog.};

\node[reg] (mar) at (3.5,0) {MAR\\= \texttt{100}};
\node[label] at (mar.north) {Reg. d'Adresse};

\node[mem] (memory) at (7,0) {Mém[\texttt{100}]\\= \texttt{ADD 200}};
\node[label] at (memory.north) {Mémoire};

\node[reg] (mdr) at (7,-2.5) {MDR\\= \texttt{ADD 200}};
\node[label] at (mdr.north) {Reg. de Données};

\node[reg] (ir) at (3.5,-2.5) {IR\\= \texttt{ADD 200}};
\node[label] at (ir.north) {Reg. d'Instruction};

\node[reg] (ac) at (0,-2.5) {AC\\= AC + 45};
\node[label] at (ac.north) {Accumulateur};

% Arrows with step numbers
\draw[arrow] (pc) -- (mar) node[midway, above, data] {\circled{1}};
\draw[arrow] (mar) -- (memory) node[midway, above, data] {\circled{2}};
\draw[arrow] (memory) -- (mdr) node[midway, right, data] {\circled{3}};
\draw[arrow] (mdr) -- (ir) node[midway, above, data] {\circled{4}};
\draw[arrow] (ir) -- (ac) node[midway, above, data] {\circled{5}};

\end{tikzpicture}
\caption{Flux de données entre les registres lors du cycle d'instruction}
\end{figure}

\section*{Exercice 6 : Explicitez le mécanisme de conversion du décimal 41.6875 en binaire
101001.1011}

Pour convertir un nombre décimal à virgule en binaire, on sépare le problème en deux parties : la partie entière (avant la virgule) et la partie fractionnaire (après la virgule).

1. La partie entière : 41

On utilise la méthode des divisions successives par 2. On note les restes de bas en haut.

\begin{itemize}
   \item \textbf{$41 \div 2 = 20$} , reste 1
   \item \textbf{$20 \div 2 = 10$} , reste 0
   \item \textbf{$10 \div 2 = 5$} , reste 0
   \item \textbf{$5 \div 2 = 2$} , reste 1
     \item \textbf{$2 \div 2 = 1$} , reste 0
   \item \textbf{$1 \div 2 = 0$} , reste 1
\end{itemize}
\noindent On obtient : $(41)_{10} = \mathbf{101001_2}$

2. La partie fractionnaire : 0.6875

Ici, on utilise des multiplications successives par 2.
\begin{itemize}
   \item \textbf{$0.6875 \times 2 = \mathbf{1}.375$ $\rightarrow$ on note 1, il reste $0.375$} 
   \item \textbf{$0.375 \times 2 = \mathbf{0}.75$ $\rightarrow$ on note 0, il reste $0.75$}
   \item \textbf{$0.75 \times 2 = \mathbf{1}.5$ $\rightarrow$ on note 1, il reste $0.5$}
   \item \textbf{$0.5 \times 2 = \mathbf{1}.0$ $\rightarrow$ on note 1, il reste $0$} 
\end{itemize}
\noindent Donc en lisant de haut en bas, on obtient : $\mathbf{.1011}$
On réunit les deux parties séparées par la virgule binaire :$$101001 + .1011 = \mathbf{101001.1011_2}$$

\newpage

\section*{Exercice 7 : Complément à $(r - 1)$ en octal et hexadécimal}

\subsection*{Rappel}

Pour un nombre $N$ en base $r$ ayant $n$ chiffres, le complément à $(r-1)$ est défini par :
$$(r^n - 1) - N$$

La méthode pratique consiste à soustraire \textbf{chaque chiffre} individuellement de $(r-1)$.

\subsection*{a) Complément à 7 d'un nombre octal}

En base octale, $r = 8$ donc $r - 1 = 7$. Le complément à 7 s'obtient en \textbf{soustrayant chaque chiffre octal de 7}.

\medskip
\textbf{Exemple :} Complément à 7 de $\mathbf{5274_8}$ \ ($n = 4$ chiffres)

$$C_7(5274_8) = (8^4 - 1) - 5274_8 = 7777_8 - 5274_8$$

\begin{center}
\renewcommand{\arraystretch}{1.3}
\begin{tabular}{|c|c|c|c|c|}
\hline
\textbf{Position} & Chiffre 4 & Chiffre 3 & Chiffre 2 & Chiffre 1 \\
\hline
$r - 1 = 7$ & 7 & 7 & 7 & 7 \\
\hline
$N$ & 5 & 2 & 7 & 4 \\
\hline
\textbf{Résultat} & \textbf{2} & \textbf{5} & \textbf{0} & \textbf{3} \\
\hline
\end{tabular}
\end{center}

$$\boxed{C_7(5274_8) = 2503_8}$$

\textbf{Vérification :} $5274_8 + 2503_8 = 7777_8 = 8^4 - 1$ \checkmark

\subsection*{b) Complément à 15 (F) d'un nombre hexadécimal}

En base hexadécimale, $r = 16$ donc $r - 1 = 15 = \text{F}_{16}$. Le complément à F s'obtient en \textbf{soustrayant chaque chiffre hexadécimal de F (15)}.

\medskip
\textbf{Exemple :} Complément à F de $\mathbf{3\text{A}7_{16}}$ \ ($n = 3$ chiffres)

$$C_F(3\text{A}7_{16}) = (16^3 - 1) - 3\text{A}7_{16} = \text{FFF}_{16} - 3\text{A}7_{16}$$

\begin{center}
\renewcommand{\arraystretch}{1.3}
\begin{tabular}{|c|c|c|c|}
\hline
\textbf{Position} & Chiffre 3 & Chiffre 2 & Chiffre 1 \\
\hline
$r - 1 = $ F (15) & F (15) & F (15) & F (15) \\
\hline
$N$ & 3 & A (10) & 7 \\
\hline
\textbf{Soustraction} & $15 - 3 = 12$ & $15 - 10 = 5$ & $15 - 7 = 8$ \\
\hline
\textbf{Résultat} & \textbf{C} & \textbf{5} & \textbf{8} \\
\hline
\end{tabular}
\end{center}

$$\boxed{C_F(3\text{A}7_{16}) = \text{C}58_{16}}$$

\textbf{Vérification :} $3\text{A}7_{16} + \text{C}58_{16} = \text{FFF}_{16} = 16^3 - 1$ \checkmark

\subsection*{Tableau récapitulatif par analogie}

\begin{center}
\renewcommand{\arraystretch}{1.5}
\begin{tabular}{|c|c|c|p{5.5cm}|l|}
\hline
\textbf{Base} & $r$ & $r-1$ & \textbf{Méthode} & \textbf{Exemple} \\
\hline
Binaire & 2 & 1 & Inverser chaque bit ($0 \leftrightarrow 1$) & $C_1(1011001) = 0100110$ \\
\hline
Octale & 8 & 7 & Soustraire chaque chiffre de 7 & $C_7(5274_8) = 2503_8$ \\
\hline
Décimale & 10 & 9 & Soustraire chaque chiffre de 9 & $C_9(546700) = 453299$ \\
\hline
Hexadécimale & 16 & F & Soustraire chaque chiffre de F & $C_F(3\text{A}7_{16}) = \text{C}58_{16}$ \\
\hline
\end{tabular}
\end{center}

\begin{figure}[H]
\centering
\begin{tikzpicture}[
    box/.style={rectangle, draw, minimum width=1.2cm, minimum height=0.8cm, text centered, font=\large},
    label/.style={font=\small\bfseries},
    arrow/.style={-latex', thick}
]

% Octal example
\node[label] at (-2, 2) {Octal :};
\node[box, fill=orange!15] (o1) at (0,2) {5};
\node[box, fill=orange!15] (o2) at (1.5,2) {2};
\node[box, fill=orange!15] (o3) at (3,2) {7};
\node[box, fill=orange!15] (o4) at (4.5,2) {4};

\node[font=\large] at (6,2) {$\rightarrow$};

\node[box, fill=green!15] (r1) at (7.5,2) {2};
\node[box, fill=green!15] (r2) at (9,2) {5};
\node[box, fill=green!15] (r3) at (10.5,2) {0};
\node[box, fill=green!15] (r4) at (12,2) {3};

\node[font=\scriptsize, above] at (0,2.5) {$7-5$};
\node[font=\scriptsize, above] at (1.5,2.5) {$7-2$};
\node[font=\scriptsize, above] at (3,2.5) {$7-7$};
\node[font=\scriptsize, above] at (4.5,2.5) {$7-4$};

% Hex example
\node[label] at (-2, 0) {Hexa :};
\node[box, fill=purple!15] (h1) at (0,0) {3};
\node[box, fill=purple!15] (h2) at (1.5,0) {A};
\node[box, fill=purple!15] (h3) at (3,0) {7};

\node[font=\large] at (6,0) {$\rightarrow$};

\node[box, fill=green!15] (s1) at (7.5,0) {C};
\node[box, fill=green!15] (s2) at (9,0) {5};
\node[box, fill=green!15] (s3) at (10.5,0) {8};

\node[font=\scriptsize, above] at (0,0.5) {$F-3$};
\node[font=\scriptsize, above] at (1.5,0.5) {$F-A$};
\node[font=\scriptsize, above] at (3,0.5) {$F-7$};

\end{tikzpicture}
\caption{Illustration du complément à $(r-1)$ : chaque chiffre est soustrait de $(r-1)$}
\end{figure}

\newpage

\section*{Exercice 8 : Configuration RAM et ROM de la carte mémoire}

\subsection*{Analyse du bus d'adresses}

Le bus d'adresses comporte \textbf{10 bits} (A10 à A1), ce qui permet d'adresser $2^{10} = 1024$ emplacements mémoire (adresses \texttt{0000} à \texttt{03FF}).

\begin{figure}[H]
\centering
\begin{tikzpicture}[scale=0.85, every node/.style={transform shape}]
    % Memory blocks
    \draw[fill=blue!15, draw=black, thick] (0,0) rectangle (5,1.5);
    \node at (2.5,0.75) {\large RAM 1 -- 128 octets};
    \node[left, font=\footnotesize\ttfamily] at (-0.2,0) {0000};
    \node[left, font=\footnotesize\ttfamily] at (-0.2,1.5) {007F};

    \draw[fill=blue!25, draw=black, thick] (0,1.5) rectangle (5,3);
    \node at (2.5,2.25) {\large RAM 2 -- 128 octets};
    \node[left, font=\footnotesize\ttfamily] at (-0.2,1.5) {0080};
    \node[left, font=\footnotesize\ttfamily] at (-0.2,3) {00FF};

    \draw[fill=blue!15, draw=black, thick] (0,3) rectangle (5,4.5);
    \node at (2.5,3.75) {\large RAM 3 -- 128 octets};
    \node[left, font=\footnotesize\ttfamily] at (-0.2,3) {0100};
    \node[left, font=\footnotesize\ttfamily] at (-0.2,4.5) {017F};

    \draw[fill=blue!25, draw=black, thick] (0,4.5) rectangle (5,6);
    \node at (2.5,5.25) {\large RAM 4 -- 128 octets};
    \node[left, font=\footnotesize\ttfamily] at (-0.2,4.5) {0180};
    \node[left, font=\footnotesize\ttfamily] at (-0.2,6) {01FF};

    \draw[fill=red!15, draw=black, thick] (0,6.2) rectangle (5,9.2);
    \node at (2.5,7.7) {\large ROM -- 512 octets};
    \node[left, font=\footnotesize\ttfamily] at (-0.2,6.2) {0200};
    \node[left, font=\footnotesize\ttfamily] at (-0.2,9.2) {03FF};

    % Braces
    \draw[decorate, decoration={brace, amplitude=10pt, mirror}] (5.3,0) -- (5.3,6)
        node[midway, right=14pt, align=left] {\textbf{RAM}\\$4 \times 128 = 512$ o};
    \draw[decorate, decoration={brace, amplitude=10pt, mirror}] (5.3,6.2) -- (5.3,9.2)
        node[midway, right=14pt, align=left] {\textbf{ROM}\\512 o};

    % Total
    \draw[decorate, decoration={brace, amplitude=12pt, mirror}] (9,0) -- (9,9.2)
        node[midway, right=16pt, align=left] {\textbf{Total}\\1024 octets\\$= 2^{10}$};
\end{tikzpicture}
\caption{Carte mémoire -- répartition de l'espace d'adressage}
\end{figure}

\subsection*{Configuration de la RAM}

La mémoire RAM est composée de \textbf{4 puces de 128 octets} chacune ($2^7$), pour un total de \textbf{512 octets} (adresses \texttt{0000} à \texttt{01FF}).

\begin{itemize}
    \item \textbf{Adressage interne :} Chaque puce utilise les bits \textbf{A7 à A1} (7 bits) pour adresser ses 128 emplacements.
    \item \textbf{Sélection de puce :} Le bit \textbf{A10 = 0} identifie la zone RAM. Les bits \textbf{A9} et \textbf{A8} sélectionnent la puce active grâce à un décodeur 2 vers 4 :
\end{itemize}

\begin{center}
\renewcommand{\arraystretch}{1.3}
\begin{tabular}{|c|c|c|c|l|}
\hline
\textbf{A10} & \textbf{A9} & \textbf{A8} & \textbf{A7--A1} & \textbf{Composant sélectionné} \\
\hline
0 & 0 & 0 & x x x x x x x & RAM 1 (\texttt{0000}--\texttt{007F}) \\
\hline
0 & 0 & 1 & x x x x x x x & RAM 2 (\texttt{0080}--\texttt{00FF}) \\
\hline
0 & 1 & 0 & x x x x x x x & RAM 3 (\texttt{0100}--\texttt{017F}) \\
\hline
0 & 1 & 1 & x x x x x x x & RAM 4 (\texttt{0180}--\texttt{01FF}) \\
\hline
1 & \multicolumn{2}{c|}{x} & x x x x x x x & ROM (\texttt{0200}--\texttt{03FF}) \\
\hline
\end{tabular}
\end{center}

\textit{Note :} Les \texttt{x} représentent des bits « libres » utilisés pour l'adressage interne de chaque composant.

\subsection*{Configuration de la ROM}

La mémoire ROM est composée d'\textbf{une seule puce de 512 octets} ($2^9$), occupant les adresses \texttt{0200} à \texttt{03FF}.

\begin{itemize}
    \item \textbf{Adressage interne :} La puce utilise les bits \textbf{A9 à A1} (9 bits) pour adresser ses 512 emplacements.
    \item \textbf{Sélection de puce :} Le bit \textbf{A10 = 1} suffit à activer la ROM.
\end{itemize}

\subsection*{Logique de sélection des puces}

\begin{figure}[H]
\centering
\begin{tikzpicture}[
    block/.style={rectangle, draw, minimum width=2.5cm, minimum height=1cm, align=center, font=\small},
    chip/.style={rectangle, draw, minimum width=2cm, minimum height=0.7cm, align=center, font=\small},
    arrow/.style={-latex', thick}
]

% A10 input
\node[font=\bfseries] (a10) at (0,3) {A10};

% Decision block
\node[block, fill=orange!15] (test) at (3,3) {A10 = ?};

% Decoder
\node[block, fill=green!15, minimum height=2cm] (dec) at (3,0) {Décodeur\\2 vers 4\\(A9, A8)};

% RAM chips
\node[chip, fill=blue!10] (r1) at (7.5, 1.5) {RAM 1};
\node[chip, fill=blue!10] (r2) at (7.5, 0.5) {RAM 2};
\node[chip, fill=blue!10] (r3) at (7.5,-0.5) {RAM 3};
\node[chip, fill=blue!10] (r4) at (7.5,-1.5) {RAM 4};

% ROM chip
\node[chip, fill=red!10, minimum height=1.2cm] (rom) at (7.5, 4) {ROM};

% A9,A8 input
\node[font=\bfseries] (a98) at (0,0) {A9, A8};

% A7-A1 input
\node[font=\bfseries] (a71) at (0,-2.5) {A7--A1};

% A9-A1 input
\node[font=\bfseries] (a91) at (0,5) {A9--A1};

% Connections
\draw[arrow] (a10) -- (test);
\draw[arrow] (test) -- node[right, font=\small] {0} (dec);
\draw[arrow] (test) -- node[above, font=\small] {1} (rom);
\draw[arrow] (a98) -- (dec);

\draw[arrow] (dec.east) ++(0, 0.6) -- (r1.west);
\draw[arrow] (dec.east) ++(0, 0.2) -- (r2.west);
\draw[arrow] (dec.east) ++(0,-0.2) -- (r3.west);
\draw[arrow] (dec.east) ++(0,-0.6) -- (r4.west);

\draw[arrow, dashed] (a71.east) -- ++(3,0) |- (r1.south west);
\draw[arrow, dashed] (a91.east) -- ++(3,0) |- (rom.south west);

% Labels
\node[font=\scriptsize\itshape, below right] at (3.2,-2.5) {adressage interne RAM};
\node[font=\scriptsize\itshape, above right] at (3.2,5) {adressage interne ROM};

\end{tikzpicture}
\caption{Logique de sélection des puces mémoire}
\end{figure}

La mémoire totale adressable est donc : $512 \text{ (RAM)} + 512 \text{ (ROM)} = \textbf{1024 octets}$, ce qui correspond bien aux $2^{10}$ combinaisons possibles du bus d'adresses de 10 bits.

\newpage

\section*{Sources}

\begin{enumerate}
    \item M. Morris Mano, \textit{Computer System Architecture}, 3\textsuperscript{e} édition, Pearson, 1993.
    \item M. Morris Mano, \textit{Digital Design}, 5\textsuperscript{e} édition, Pearson, 2012.
    \item William Stallings, \textit{Computer Organization and Architecture}, 10\textsuperscript{e} édition, Pearson, 2016.
    \item Andrew S. Tanenbaum, \textit{Structured Computer Organization}, 6\textsuperscript{e} édition, Pearson, 2012.
    \item Notes de cours INF101 -- Introduction aux outils informatiques, ISTEAH, Hiver 2026.
    \item Images des exercices 1 et 2 générées par IA (Amazon Titan Image Generator v2).
\end{enumerate}

\end{document}

